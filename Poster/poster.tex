\documentclass[a0,portrait]{a0poster}
\special{papersize=91.44cm,121.92cm}
\setlength{\paperwidth}{91.44cm}
\setlength{\paperheight}{121.92cm}
\usepackage[spanish]{babel}
\usepackage[utf8]{inputenc}
\usepackage{anyfontsize}
\usepackage{color}
\usepackage{multicol}
\columnsep=100pt
\def\changemargin#1#2{\list{}{\rightmargin#2\leftmargin#1}\item[]}
\let\endchangemargin=\endlist 
\usepackage[percent]{overpic}
\topskip0pt
\usepackage[top=0.15in, bottom=0in, left=-0.5in, right=0in]{geometry}
\begin{document}
\begin{overpic}[width=92.54cm,height=8.26cm]{encabezado}
\end{overpic}
\vspace{3cm}

\begin{changemargin}{3cm}{3cm} 
\begin{center}
{\fontsize{80}{80}\selectfont \textbf{\textcolor{blue}{PYTHON TOOL (MIFSA) FOR ANALYSIS OF
INTEGRAL FIELD SPECTRA}}}
\vspace{2cm}
{\center \fontsize{60}{40}\selectfont \textbf{Jiménez-Cárdenas, Mario-A Higuera-G}}
\vspace{2cm}
{\center \fontsize{60}{40}\selectfont Physics Department, National University of Colombia; National Astronomical Observatory}
\end{center}
\vspace{2cm}
\begin{multicols*}{3}

\fbox{%
    \parbox{\columnwidth}{
\fontsize{40}{40} \selectfont \textbf{\textcolor{red}{Introduction}}\\ \\
We developed the code named MIFSA (Map Integral Field Spectroscopy Analysis) in Python, which makes the extraction  of the data-cubes of
the galaxies contained in the database CALIFA
(Calar Alto Legacy Integral Field Spectroscopy Area
Survey). The code allows to represent the image
of the galaxy, calculating the total flux associated
to each pixel, normalizing all fluxes and making
a spectral colormap with the normalized values
corresponding to each pixel.
The program uses the library PyFits (from
Astropy) to read datacubes, Tkinter for graphic
interface, Matplotlib to show the galaxy and its
spectra, and Numpy to work with arrays. Despite
that there are some tools in some languages, MIFSA
is a free Python alternative for the analysis of
integral field spectra.
In order to analize datacube’s information,
MIFSA can extract and show the spectra associated
to each pixel and identify its absortion and emission
lines.
 }%
}
\\ \\ \\ \\
%==================================
\fbox{%
    \parbox{\columnwidth}{
    	\fontsize{40}{40} \selectfont \textbf{\textcolor{red}{Aim}}\\ \\
    	There are plenty of tools developed (or under development) that do the basics of an analysis integral field spectra. The most developed (PINGSOFT) was written in IDL, so is neccesary to buy the license to use it. Other ones are 
    }
}
\end{multicols*}
\end{changemargin}
\end{document}